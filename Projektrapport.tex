\documentclass[10pt,a4paper]{article}
\usepackage[utf8]{inputenc}
\usepackage{amsmath}
\usepackage{amsfonts}
\usepackage{amssymb}
\usepackage{graphicx}
\usepackage[english]{babel}
\usepackage[utf8]{inputenc}
\usepackage{amsmath}

\graphicspath{}

\author{
  \texttt{Theodor Fleming}
  \and
  \texttt{Andreas Nordberg}
}

\begin{document}
\pagenumbering{gobble}

\title{TDDD66 Projekt}
\maketitle

\cleardoublepage

\begin{abstract}
This project explains what multicast, used by WiFi, is, when, how and where it is used and how you as a consumer can take advantage of this knowledge to increase the speed of your home WiFi. The report compares how similar services work and differentiate from each other. It will also briefly cover how the security works when using multicast, and the measures used to block out unwanted receivers of the shared data.
Our conclusion after doing research and performed different tests is that there are some minor things you can do at home to enhance the performance of your WiFi. One way of doing this would be to disconnect the lowest common denominator, the device which slows the network down. 

(To be revised)
\end{abstract}

\newpage

\section{Inledning}

\subsection{Context}

Multicasting is a method of sharing information by communicating with a select amount of receivers. Other methods exist for this purpose, such as unicast and broadcast, but these will play a lesser role in this study and are only to be compared with multicasting. There are lots of different kinds of networks that uses multicasting to distribute information, but this paper will only focus on how multicasting is handled with WiFi.

As with any means of solving a problem, different results can be expected from solving the problem using different methods. In our case with this report, how does WiFi multicasting compare to other means of sharing information within a network in terms of different methods of routing? How well does multicasting do compared to other methods of routing in terms of performance? These are the kind questions this report will delve into and try to answer.  

\subsection{Motivation}

This report will cover why multicasting is a slower service than other communication services of the sort, for example unicast. The following questions then are, what can be done to reduce the discrepancy in speed between the routing methods? How can the throughput in a network be improved with multicast in relation to the other methods of routing.

This seems to be an important topic since WiFi data traffic is commonly distributed using multicasting. Our hypothesis is that by increasing the efficiency of how multicasting is handled, the overall speed of the WiFi network would increase.

\subsection{Narrow Scope}

In this study we will do tests and experiments with multicasting using a number of different devices and recievers too see how the speed is altered. Apparently, by using multicasting, the slowest data rate that the slowest reciever can operate with is used and we want to perform experiments on how we can control and change the net speed in the network. Perhaps it is better to disconnect your deprecated phone from your home WiFi network when you aren't using it to not slow down the network using multicast? 
Our main goal is to see if you as a private user can get faster WiFi by manipulating the speed, options and use of multicast. We hope that the result of this study will help WiFi users how to maximize the speed of your home WiFi network.

\subsection{Methodology}

We will compare the WiFi speed when using different devices, options and recievers. We will also compare the WiFi speed when using multicast and unicast too see if there is a big difference. To achieve this, we will be using some kind of WiFi speed measure application to be able to see what speed the WiFi is operating on and if we can increase the it somehow. 

\subsection{Gameplan}

We are going to do research to understand more how multicast is used, how it works and how we can manipulate it until milestone 2. We need to understand and get a better feel for multicast before we can start with our tests. After that we are going to look into what we can experiment with until milestone 3 and fill the report with more fact about multicast used with Wi-Fi, pros, cons and comparison with other communication services. Now we are going to prepare for the seminar presentation, get feedback and fix the report until final report.

\newpage

\section{Background}

\subsection{Theory}

\subsection{Experiment}

\newpage

\section{Resultat}

\subsection{Presentation of results}

\subsection{Discussing results}

\end{document}